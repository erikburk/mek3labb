\documentclass[12pt,a4paper]{article}
\usepackage[utf8]{inputenc}
\usepackage[hidelinks]{hyperref}
\usepackage[swedish]{babel}
\usepackage[T1]{fontenc}
\usepackage[parfill]{parskip} 
\usepackage[a4paper, total={6in,8in}]{geometry}
\usepackage{mathtools,graphicx,url,amsmath, tikz, csquotes, biblatex, upgreek, newtxtext, newtxmath, tcolorbox, wrapfig, physics}

% Gör en mapp som heter bilder och lägg sedan in alla bilder där för att använda nedan.
%\graphicspath{{./bilder/}}

% Numrera i varje sektion, kommentera ut för att bli av med det.
\numberwithin{equation}{section}
\numberwithin{figure}{section}
\newcounter{excount}
\counterwithin{excount}{section}
\setcounter{excount}{0}

%annat
%\newcommand{\abs}[1]{\left\lvert #1 \right\rvert }
\newcommand{\unit}[1]{\textrm{ #1}}
\newcommand{\der}[2]{\frac{d#1}{d#2}}
\newcommand{\pd}[2]{\frac{\partial #1}{\partial #2}}
\newcommand{\sgn}{\, \textrm{sgn}}

%enhetsvektorer
\newcommand{\iunit}{\, \hat{i}}
\newcommand{\junit}{\, \hat{j}}
\newcommand{\kunit}{\, \hat{k}}

%lådor för definition, exempel och notering.
\newcommand{\exempel}[2]{
    \begin{tcolorbox}[colback=white,colframe=blue!80,coltitle=white,title={Exempel \stepcounter{excount} \theexcount : #1}]
        #2
    \end{tcolorbox}
}
\newcommand{\notera}[1]{
    \begin{tcolorbox}[colback=red!5!white,colframe=red!75,coltitle=white,title= Notera att:]
        #1
    \end{tcolorbox}
}
\newcommand{\defin}[2]{
    \begin{tcolorbox}[colback=yellow!5!white,colframe=yellow!100,coltitle=black,title= Definition: #1]
        #2
    \end{tcolorbox}
}
\newcommand{\sats}[2]{
    \begin{tcolorbox}[colback=green!5!white,colframe=green!75,coltitle=black,title= Sats: #1]
        #2
    \end{tcolorbox}
}

%figurer, använda bara vanliga lol detta var inte värt det
\newcommand{\figil}[2]{
    \begin{center}
        \includegraphics[width=#1\textwidth]{#2}
    \end{center}
}
\newcommand{\figfl}[5]{
    \begin{figure}[#1]
        \centering
        \includegraphics[width=#2\textwidth]{#3}
        \caption{#4}
        \label{fig:#5}
    \end{figure}
}
\newcommand{\wf}[5]{
    \begin{wrapfigure}{#1}{#2\textwidth}
        \includegraphics[width=#2\textwidth]{#3}
        \caption{#4}
        \label{fig:#5}
    \end{wrapfigure}}

% kommentera ut för att använda.
%\addbibresource{referenser.bib}

\title{Mekanik III - Labbrapport}
\author{Erik Björk, Jakob Dahlgren}
\date{\today}

\begin{document}
\maketitle
\clearpage

\tableofcontents

\begin{abstract}
    blah
\end{abstract}

\section{Inledning}
Laborationen undersökte kopplade svängningar hos två pendlar, med en fjäder emellan. Det jämfördes med teoretiska approximationer, där systemets Lagrangian beräknades analytiskt. Vidare undersöktes hur koppling påverkar oscillerande system, där energi överfördes mellan pendlarna.
\section{Teori}

Kopplade svängningar löses generellt i fyra steg:
\begin{enumerate}
    \item Ställ upp Lagrangianen, antag små svängningar, och lös för rörelseekvationerna.
    \item gör en ansats: \begin{equation}
        q_j = A_j e^{i\omega t},
    \end{equation}
    för $j$ generaliserade koordinater.
    
    \item Lös matrisekvationen som uppstår för att hitta vinkelfrekvenserna $\omega$, samt hitta relationerna mellan koefficienter. Detta ger den allmänna lösningen.
    \item Använd begynnelsevillkor för att få ut den fullständiga lösningen.
\end{enumerate}
Vidare behövdes också mätningar och beräkningar av stångens tröghetsmoment, massa, samt fjäderkonstanten.

\subsection{Lösning av Systemet}

\begin{figure}[htbp]
    \centering
    placeholder
    \caption{Uppställning av systemet.}
    \label{fig:uppställning}
\end{figure}

Utgående från uppställningen i figur \ref{fig:uppställning} har man två tunna stänger med en fjäder emellan i punkterna $P_1,P_2$. Tillsammans med objektet längst ut har pendlarna ett tröghetsmoment kring $O$ som $I_0$, och ett masscentrum i $C_1,C_2$. Eftersom pendlarna inte har någon rörelse relativt $O$ så blir den totala kinetiska energin \begin{equation}
    T = \frac{1}{2}I_0 \dot{\phi}^2 + \frac{1}{2}I_0 \dot{\theta}^2 = \frac{I_0}{2}(\dot{\phi}+\dot{\theta})
\end{equation}
Den potentiella energin ges från gravitationen och fjädern. För det krävs positionerna av punkterna.
\begin{align}
    \vb r_{P1} &= l (\sin\phi\iunit-\cos\phi\junit)\\
    \vb r_{P2} &= l (\sin\theta\iunit-\cos\theta\junit)\\
    \vb r_{C1} &= L (\sin\phi\iunit-\cos\phi\junit)\\
    \vb r_{C2} &= L (\sin\theta\iunit-\cos\theta\junit)\\
\end{align}
relativt respektive punkt $O$ för varje pendel. Därefter kan gravitationspotentialen skrivas som \begin{align}
    U_W &= mg \vb r_{C1}(1-\cdot \junit) + mg \vb(1- r_{C2}\cdot \junit)\\
    &=mgL(2-\cos\phi-\cos\theta)
\end{align}
Fjädern antas ha jämviktsläge när pendlarna är rakt ner, det vill säga när vinklarna är 0. Därmed blir \begin{align}
    U_{\textrm{fjäder}} &= \frac{1}{2}k(\vb r_{P1}-\vb r_{P2})^2 \\ &= \frac{kl^2}{2}(\sin^2\theta+\sin^2\phi+\cos^2\theta+\cos^2\phi-\sin\theta\sin\phi-\cos\theta\cos\phi)\\
    &=\frac{kl^2}{2}\cos(\theta-\phi)
\end{align}
Tillsammans blir den potentiella energin då \begin{equation}
    U = kl^2(1-\cos(\theta-\phi)) + mgL(2-\cos\phi-\cos\theta)
\end{equation}
Vilket ger Lagrangianen \cite{nordling_physics_2021} \begin{equation}
    L = T-U = \frac{I_0}{2}(\dot{\phi}^2+\dot{\theta}^2) - kl^2(1\cos(\theta-\phi)) -mgL(2-\cos\phi-\cos\theta)
\end{equation}
Redan här antas systemet ha små svängningar. Termerna med $\sin,\cos$ approximeras till ledande ordning 2. Det innebär att \begin{equation}
    \sin x = x,\quad \cos x = 1-\frac{x^2}{2}
\end{equation}
med högre ordningens termer försummade, enligt funktionernas Taylorutveckling \cite{nordling_physics_2021}. Det innebär att Lagrangianen blir \begin{equation}
    L = \frac{I_0}{2}(\dot{\phi}^2+\dot{\theta}^2)+ \frac{kl^2}{2}(\theta-\phi)^2-\frac{mgL}{2}(\theta^2+\phi^2)
\end{equation}
Rörelseekvationerna blir då \begin{align}
    \der{}{t}\pd{L}{\dot{\theta}}-\pd{L}{\theta} = 0 = I\ddot{\theta}+mgL\theta+kl^2(\theta-\phi) \label{eq:rörekv theta}\\
    \der{}{t}\pd{L}{\dot{\phi}}-\pd{L}{\phi} = 0 = I\ddot{\phi}+mgL\phi+kl^2(\phi-\theta) \label{eq:rörekv phi}
\end{align}
Detta ser ut som kopplade svängningar. För att lösa systemet görs ansatserna \begin{equation}
    \begin{pmatrix}
        \theta\\\phi
    \end{pmatrix} = \begin{pmatrix}
        A\\B
    \end{pmatrix}e^{i\omega t}
\end{equation}
Vilket insatt i (\ref{eq:rörekv theta}), (\ref{eq:rörekv phi}) (samt förkortat med exponentialen) ger \begin{align}
    \begin{cases}
        0 = (-I\omega^2+mgL+kl^2)A-{kl^2}B \\
        0 = (-I\omega^2+mgL+kl^2)B-{kl^2}A
    \end{cases}
    \label{eq:rörelse efter ansats}
\end{align}
Vilket är ekvivalent med det linjära ekvationssystemet \begin{equation}
    \begin{pmatrix}
        -I\omega^2+mgL+kl^2 & -kl^2 \\
        -kl^2  &-I\omega^2+mgL+kl^2 
    \end{pmatrix} \begin{pmatrix}
        A\\B
    \end{pmatrix} = 0
\end{equation}
Söker icke-triviala lösningar. Det ges när determinanten av matrisen är 0, vilket innebär att \begin{align}
    \det \begin{pmatrix}
        -I\omega^2+mgL+kl^2 & -kl^2 \\
        -kl^2  &-I\omega^2+mgL+kl^2 
    \end{pmatrix} &= 0 \\
    (-I\omega^2+mgL+kl^2)^2-(kl^2)^2 &= 0\\
    I^2\omega^4-2I\omega^2(mgL+kl^2)+(mgL+kl^2)^2-k^2l^4 &=0\\
    %I^2\omega^4+2I\omega^2(mgL+kl^2)+(mgL)^2+2mgLkl^2+k^2l^4-k^2l^4&=0\\
    %I^2\omega^4+2I\omega^2(mgL+kl^2)+(mgL)^2+2mgLkl^2 &= 0
\end{align}
Använder PQ-formeln för att lösa för $\omega$:
\begin{align}
    \omega^2 &= mgL+kl^2 \pm \frac{1}{I}\sqrt{ (mgL+kl^2)^2-(mgL+kl^2)^2 + k^2l^4 }\\
    &= \frac{mgL+kl^2\pm kl^2}{I} 
\end{align}
Det vill säga, systemets egenfrekvenser är \begin{equation}
    \omega_1^2 = \frac{mgL}{I}, \quad  \omega_2^2 = \frac{mgL+2kl^2}{I}
\end{equation}
Vilket ger den allmänna lösningen 
\begin{equation}
    \begin{cases}
        \theta(t) = A_1^+e^{i\omega_1t}+A_1^-e^{-i\omega_1t}+A_2^+e^{i\omega_2t}+A_2^-e^{-i\omega_2t}\\
        \phi(t) = B_1^+e^{i\omega_1t}+B_1^-e^{-i\omega_1t}+B_2^+e^{i\omega_2t}+B_2^-e^{-i\omega_2t}
    \end{cases}
\end{equation}
Denna lösning har dubbelt så många konstanter som förväntat för två stycken andra ordningens differentialekvationer. Beroendet mellan konstanterna hittas genom insättning i (\ref{eq:rörelse efter ansats}). Börjar med $\omega_1$:
\begin{equation}
    \begin{cases}
        0 = (-mgL+mgL+kl^2)A_1^\pm-{kl^2}B_1^\pm \\
        0 = (-mgl+mgL+kl^2)B_1^\pm-{kl^2}A_1^\pm
    \end{cases}
\end{equation}
Vilket ger att \begin{equation}
    A_1^\pm = B_1^\pm
\end{equation}
Därefter, $\omega_2$: \begin{equation}
    \begin{cases}
        0 = (-mgL-2kl^2+mgL+kl^2)A_2^\pm-{kl^2}B_2^\pm \\
            0 = (-mgl-2kl^2+mgL+kl^2)B_2^\pm-{kl^2}A_2^\pm
    \end{cases}
\end{equation}
vilkt innebär att \begin{equation}
    A_2^\pm = -B_2^\pm
\end{equation}
och den allmänna lösningen reduceras till \begin{equation}
    \begin{cases}
        \theta(t) = A_1^+e^{i\omega_1t}+A_1^-e^{-i\omega_1t}+A_2^+e^{i\omega_2t}+A_2^-e^{-i\omega_2t}\\
        \phi(t) = A_1^+e^{i\omega_1t}+A_1^-e^{-i\omega_1t}-A_2^+e^{i\omega_2t}-A_2^-e^{-i\omega_2t}
    \end{cases}
\end{equation}
%\printbibliography
Genom kombinering kan då systemets normalkoordinater skapas: \begin{align}
    q_1(t) &= \theta(t)+\phi(t) = 2A_1^+ e^{i\omega_1t} +2A_1^-e^{i\omega_1 t}\\
    q_2(t) &= \theta(t)-\phi(t) = 2A_2^+ e^{i\omega_2t} +2A_2^-e^{i\omega_2 t}
\end{align}
Vilka uppfyller ekvationen för harmoniska oscillatorer med respektive egenfrekvens: \begin{equation}
    \ddot{q}_i+\omega_i^2 q_i = 0
\end{equation}
Omvandling tillbaka till $\theta,\phi$ går man baklänges: \begin{align}
    \theta(t) = \frac{1}{2}(q_1+q_2)\\
    \phi(t) = \frac{1}{2}(q_1-q_2)
\end{align}
Eftersom lösningen beskriver ett fysikaliskt och verkligt system söks reella lösningar för alla tidssteg. Utveckling av exponentialen ger då att \begin{equation}
    q_i(t) = 2\left( \cos(\omega_i t)(A_i^++A_i^-)+i\sin(\omega_i t)(A_i^+-A_i^-) \right),
\end{equation}
vilket innebär att konstanterna måste vara varandras komplexa konjugat. Notera att $i$ betecknar både index och komplexa talet ovan. Det ger oss nya definitioner av $q_1,q_2$: 
\begin{align}
    q_1 = C_{11}\sin \omega_1 t + C_{12} \cos \omega_1 t \label{eq:q1}\\
    q_2= C_{21}\sin \omega_2 t + C_{22}\cos \omega_2 t\label{eq:q2}
\end{align}
där
\begin{equation}
    C_{i1} = 2(A_i^+ + A_i^-),\quad C_{i2} = 2(A_i^+-A_i^-)
\end{equation}

\subsubsection{Fall A}
Här släpps båda pendlarna med samma vinkel $\alpha$ åt samma håll. Det innebär alltså att fjädern inte har någon påverkan. Därmed sker endast oscillation i frekvens $\omega_1$. Rörelseekvationen beskrivs då endast av $q_1(t)$. Insättning av begynnelsevillkoret ger då att \begin{equation}
    \theta(0)=\phi(0) = C_{12} = \alpha,
\end{equation}
det vill säga, det blir en ren cosinusrörelse i båda pendlarna.

\subsubsection{Fall B}
I detta fallet släpps det från motsatt håll, vilket innebär att egenfrekvensen $\omega_2$ bör vara enda bidraget:
\begin{equation}
    \theta(t)=-\phi(t) = q_2(t)
\end{equation}
Insättning av begynnelsevärde ger då igen att \begin{equation}
    \theta(0)=\alpha=-\phi(0) = C_{22}
\end{equation}
Igen, en ren cosinusrörelse i systemet.

\subsubsection{Fall C}
I fall C är pendlarnas vinklar i en superpoosition av egenfunktionerna. Begynnelsevärdena är $\theta(0)=0, \phi(0)=-\alpha$. Används uttrycken för $\theta,\phi$ från ekvation \ref{eq:q1},\ref{eq:q2},  får man då att \begin{equation}
    \begin{cases}
        \frac{1}{2}(C_{12}-C_{22})&=-\alpha\\
        \frac{1}{2}(C_{12}+C_{22})&=0
    \end{cases},
\end{equation}
vilket har den unika lösningen \begin{equation}
    C_{12} = -\alpha, \quad C_{22} = \alpha
\end{equation}

\subsection{Beräkning av Tröghetsmoment och Egenfrekvenser}
Tröghetsmomentet för pendeln behövdes för att beräkna systemets egenfrekvenser. Den består av en stång, som antas vara lång och smal, samt en massa som antas vara en ihålig cylinder. Då ges tröghetsmomenten runt kropparnas masscentrum av \cite{nordling_physics_2021}
\begin{align}
    I_{\textrm{stång}} &= \frac{1}{12}mL^2\\
    I_\textrm{cyl} &= LÄGG IN!!!!
\end{align}
Kombinerat med parallelaxelteoremet blir det totala tröghetsmomenten kring $O$ då 
\begin{equation}
    I_0 = LÄGG IN OCKSÅ!!!!
\end{equation}

\subsection{Felberäkningar}
\section{Metod}
\subsection{Materiel}
\begin{itemize}
    \item Stång med massa
    \item Fjäder
    \item Vinkelgivare, kopplad till mätprogram (2 st)
\end{itemize}

\subsection{Utförande}
Stängerna med massor hängdes upp i vinkelgivare. En fjäder sattes fast mellan stängerna. Mätprogrammet statades, och de teoretiskt härledda egenmoderna togs fram. Massorna positionerades motsvarigt till varje randvillkor, och systemets rörelse mättes. Experimentet upprepades flera gånger för varje randvillkor för att använda den mest representativa mätserien.

Efter mätningarna gjordes kurvanpassningar till mätdatan för att hitta systemets egenfrekvenser. Dessa jämfördes med den teoretiskt beräknade värdenna, baserat på mätningar av pendelns tröghetsmoment, massa och fjäderkonstanten.
\section{Resultat}

\subsection{Experimentkörning}
I figurerna \ref{fig:A},\ref{fig:B},\ref{fig:C} syns datan för de första fem sekundrarna av mätningar. Där är båda pendlarna med, som $\theta$ och $\phi$, samt egenmoderna $q_1,q_2$.

\begin{figure}[htbp]
    \centering
    \includegraphics[width=0.7\textwidth]{plot_A.pdf}
    \caption{Mätdata från fall A.}
    \label{fig:A}
\end{figure}

\begin{figure}[htbp]
    \centering
    \includegraphics[width=0.7\textwidth]{plot_C.pdf}
    \caption{Mätdata från fall C.}
    \label{fig:B}
\end{figure}

\begin{figure}[htbp]
    \centering
    \includegraphics[width=0.7\textwidth]{plot_B.pdf}
    \caption{Mätdata från fall C.}
    \label{fig:C}
\end{figure}

Från en sinusanpassning i \emph{Capstone} fick man ut egenfrekvenserna för vardera fall:
\begin{align}
    \begin{cases}
        \omega_{A1} &= 4.52\unit{rad/s}\\
        \omega_{B1} &= 4.52\unit{rad/s}\\
        \omega_{C1} &= 4.53\unit{rad/s}
    \end{cases},\quad 
    \begin{cases}
        \omega_{A2} &= 5.05\unit{rad/s}\\
        \omega_{A2} &= 5.06\unit{rad/s}\\
        \omega_{A2} &= 5.06\unit{rad/s}
    \end{cases} \label{eq: omega cap}
\end{align}

\subsection{Mätningar}
lägg in alla mätningar med osäkerheter här!!!!!
\include{diskussion}
\include{slutsats}

\appendix
\include{härledning}
\end{document}
