\section{Metod}
\subsection{Materiel}
\begin{itemize}
    \item Stång med massa
    \item Fjäder
    \item Vinkelgivare, kopplad till mätprogram (2 st)
\end{itemize}

\subsection{Utförande}
Stängerna med massor hängdes upp i vinkelgivare. En fjäder sattes fast mellan stängerna. Mätprogrammet statades, och de teoretiskt härledda egenmoderna togs fram. Massorna positionerades motsvarigt till varje randvillkor, och systemets rörelse mättes. Experimentet upprepades flera gånger för varje randvillkor för att använda den mest representativa mätserien.

Efter mätningarna gjordes kurvanpassningar till mätdatan för att hitta systemets egenfrekvenser. Dessa jämfördes med den teoretiskt beräknade värdenna, baserat på mätningar av pendelns tröghetsmoment, massa och fjäderkonstanten.